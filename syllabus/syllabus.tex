%% ****** Start of file template.aps ****** %
%%
%%
%%   This file is part of the APS files in the REVTeX 4 distribution.
%%   Version 4.0 of REVTeX, August 2001
%%
%%
%%   Copyright (c) 2001 The American Physical Society.
%%
%%   See the REVTeX 4 README file for restrictions and more information.
%%
%
% This is a template for producing manuscripts for use with REVTEX 4.0
% Copy this file to another name and then work on that file.
% That way, you always have this original template file to use.
%
% Group addresses by affiliation; use superscriptaddress for long
% author lists, or if there are many overlapping affiliations.
% For Phys. Rev. appearance, change preprint to twocolumn.
% Choose pra, prb, prc, prd, pre, prl, prstab, or rmp for journal
%  Add 'draft' option to mark overfull boxes with black boxes
%  Add 'showpacs' option to make PACS codes appear
%  Add 'showkeys' option to make keywords appear
\documentclass{revtex4}
%\documentclass[aps,prl,preprint,superscriptaddress]{revtex4}
%\documentclass[aps,prl,twocolumn,groupedaddress]{revtex4}
\usepackage[dvipdf]{graphicx}
%\usepackage{dcolumn}

% You should use BibTeX and apsrev.bst for references
% Choosing a journal automatically selects the correct APS
% BibTeX style file (bst file), so only uncomment the line
% below if necessary.
%\bibliographystyle{apsrev}

\begin{document}

% Use the \preprint command to place your local institutional report
% number in the upper righthand corner of the title page in preprint mode.
% Multiple \preprint commands are allowed.
% Use the 'preprintnumbers' class option to override journal defaults
% to display numbers if necessary
%\preprint{}

%Title of paper
\title{Phys 2502W Course Syllabus: Laboratory in Electromagnetism and Mechanics II}

% repeat the \author .. \affiliation  etc. as needed
% \email, \thanks, \homepage, \altaffiliation all apply to the current
% author. Explanatory text should go in the []'s, actual e-mail
% address or url should go in the {}'s for \email and \homepage.
% Please use the appropriate macro foreach each type of information

% \affiliation command applies to all authors since the last
% \affiliation command. The \affiliation command should follow the
% other information
% \affiliation can be followed by \email, \homepage, \thanks as well.
\author{Professor Richard Jones}
%\homepage[]{Your web page}
%\thanks{}
%\altaffiliation{}
\affiliation{Physics 2502, Spring Semester 2015, University of Connecticut}
%\author{R.T. Jones}
%\affiliation{University of Connecticut}

%Collaboration name if desired (requires use of superscriptaddress
%option in \documentclass). \noaffiliation is required (may also be
%used with the \author command).
%\collaboration can be followed by \email, \homepage, \thanks as well.
%\collaboration{}
%\noaffiliation

\date{January 23, 2015}

% insert suggested PACS numbers in braces on next line
%\pacs{}
% insert suggested keywords - APS authors don't need to do this
%\keywords{}

\setlength{\topmargin}{0in}

%\maketitle must follow title, authors, abstract, \pacs, and \keywords
\maketitle

% body of paper here - Use proper section commands
% References should be done using the \cite, \ref, and \label commands

%% The normal text is displayed in two-column format, but special
%% sections spanning both columns can be inserted within the page
%% format so that long equations can be displayed. Use
%% sparingly.
%%\begin{widetext}
%% put long equation here
%%\end{widetext}
%
%% figures should be put into the text as floats.
%% Use the graphics or graphicx packages (distributed with LaTeX2e)
%% and the \includegraphics macro defined in those packages.
%% See the LaTeX Graphics Companion by Michel Goosens, Sebastian Rahtz,
%% and Frank Mittelbach for instance.
%%
%% Here is an example of the general form of a figure:
%% Fill in the caption in the braces of the \caption{} command. Put the label
%% that you will use with \ref{} command in the braces of the \label{} command.
%% Use the figure* environment if the figure should span across the
%% entire page. There is no need to do explicit centering.
%
%%\begin{turnpage}
%% Surround figure environment with turnpage environment for landscape
%% figure
%% \begin{turnpage}
%% \begin{figure}
%% \includegraphics{}%
%% \caption{\label{}}
%% \end{figure}
%% \end{turnpage}
%
%% tables should appear as floats within the text
%%
%% Here is an example of the general form of a table:
%% Fill in the caption in the braces of the \caption{} command. Put the label
%% that you will use with \ref{} command in the braces of the \label{} command.
%% Insert the column specifiers (l, r, c, d, etc.) in the empty braces of the
%% \begin{tabular}{} command.
%% The ruledtabular enviroment adds doubled rules to table and sets a
%% reasonable default table settings.
%% Use the table* environment to get a full-width table in two-column
%% Add \usepackage{longtable} and the longtable (or longtable*}
%% environment for nicely formatted long tables. Or use the the [H]
%% placement option to break a long table (with less control than 
%% in longtable).
%
%
%% Surround table environment with turnpage environment for landscape
%% table
%% \begin{turnpage}
%% \begin{table}
%% \caption{\label{}}
%% \begin{ruledtabular}
%% \begin{tabular}{}
%% \end{tabular}
%% \end{ruledtabular}
%% \end{table}
%% \end{turnpage}
%
%% Specify following sections are appendices. Use \appendix* if there
%% only one appendix.
%%\appendix
%%\section{}
%

\section{Lecture and Lab Schedule}

\begin{itemize}
\item
Lectures will be held in the Physics Building (PB), room 303,
on Monday and Wednesday from 1:00-1:50 PM.
\item
Labs will be held in the same room immediately following the
Wednesday lectures, and last until 5:00 on Wednesday afternoons.
\end{itemize}

\section{Content and preparation}

\begin{itemize}
\item
PHYS 2502 is a laboratory course devoted to the study of
mechanical and electromagnetic phenomena, with an emphasis on
the analysis and interpretation of experimental data, and their
dissemination in the form of written scientific articles and
oral presentations.
\item
The course has a significant writing component.  A written
report describing the methods used and results obtained is required
for each experiment.  For two of the experiments, students must write
an 8-page scientific article in which they present a rationale for
the measurement, a description of the experimental setup and methods
used, the quantitative results obtained, and their interpretation.
Special attention must be given to experimental errors, and the
soundness of conclusions that can be drawn from the measurement.
\item
Students should have already successfully completed Physics 2501, in
addition to 1201 or 1401 or 1501 or 1601.  The second part of the
introductory physics sequence covering electricity and magnetism would be
helpful but is not required.  English 1010 or 1011 or 3800 are also required.
\item
PHYS 2501W is a laboratory course devoted to the study of mechanical and
electromagnetic phenomena, with an emphasis on the analysis and
interpretation of experimental data, and their dissemination in the
form of written scientific articles and oral presentations.
\item
The course has a significant writing component. A written report describing
the methods used and results obtained is required for each experiment.
For two of the experiments, students must write an 8-10 page scientific
article in which they present a rationale for the measurement, a description
of the experimental methods used, the data obtained and their analysis, and
the conclusions that can be drawn from the measurement. Contrasted with
lower-level courses with a lab, special attention must be given to
experimental errors, and their propagation to uncertainties on the final
results. Students will receive instruction in scientific writing as an
integral part of the course, as well as individual feedback from the
instructors aimed at helping them to improve their technical writing skills.
\item
Students should have already successfully completed Physics 1201 or 1401 or
1501 or 1601. The second part of the introductory physics sequence covering
electricity and magnetism would be helpful but is not required. English 1010
or 1011 or 3800 are also required.
\item
The goals for the course are that students demonstrate their ability to:
\begin{enumerate}
\item
collect experimental data</b> using computer-
controlled scientific apparatus;
\item
create an appropriate model</b> of an experimental
setup and generate predictions based on sound theoretical
principles;
\item
develop skills in numerical data analysis</b> using
appropriate applications and/or programming frameworks;
\item
understand experimental errors</b> and how they
propagate through the theoretical analysis;
\item
draw sound scientific conclusions</b> from
experimental results.
\item
write up experimental results</b> in a form suitable
for publication in a scientific journal.
\end{enumerate}
\end{itemize}

\section{Course materials}

\begin{itemize}
\item
{\bf Text books:} Students are encouraged to obtain a copy of either one or
the other of the following two books on error analysis. Taylor is a more
introductory text, with helpful examples of how data analysis techniques
are applied in practical cases. Bevington is more advanced and will be useful
as a reference for students interested in pursuing experimental research
beyond this course. Both of these books are available from online vendors
such as amazon.com.
\begin{enumerate}
\item
J.R. Taylor, {\it An Introduction to Error Analysis: The Study of Uncertainties
in Physical Measurements} (2nd ed.), University Science Books, 1997, 
ISBN 0-935702-75-X.
\item
P.R. Bevington and D.K. Robinson, {\it Data Reduction and Error Analysis for
the Physical Sciences} (3rd ed.), McGraw-Hill, 2003, ISBN 0-07-247227-8 
(see also \url{http://www.mhhe.com/bevington}).
\end{enumerate}
The following books are useful references on the subject of scientific writing.
\begin{enumerate}
\item
{AIP Style Manual} (4th ed.), American Institute of Physics, 1990, available
in pdf format at \url{http://www.aip.org/pubservs/style/4thed/AIP_Style_4thed.pdf}.
\item
R.A. Day, {\it Scientific English: A Guide for Scientists and Other 
Professionals}, Oryx, 1995, ISBN 0-89774-989-8.
\end{enumerate}
\item
{\bf Course web site:} Online access to laboratory guides, individual student
grade records, and various other useful reference materials related to the
course are available on the course web site at \url{http://learn.uconn.edu}.
Students should log in using their netid and password, and click on the link
to PHYS 2502 under their list of classes.
\end{itemize}

\section{Assignments}

\begin{itemize}

\item
{\bf In-class assignments:} In-class time is divided between two hour-long
weekly periods with the instructor and one weekly laboratory period supervised
by a teaching assistant. In-class time will be devoted to three major topics:
\begin{enumerate}
\item
introduction and theoretical background for the experiments,
\item
tutorials and group exercises in data and error analysis, and
\item
instruction in scientific writing and peer critique.
\end{enumerate}
Participation in these periods will be important to your success in the course.

\item
{\bf Regular reports:} Each student must separately submit an
individual report and abstract for each of the laboratory experiments.
In addition to the experiments, students will also be given tutorials in
advanced data analysis that they will work on in groups.  For the tutorials,
the lab instructor will accept a completed project file from each student
in place of a written report.  In all other cases, a written report is
required from each student for each experiment.  The reports are typically
4-6 pages, and must contain all of the following elements:
\begin{enumerate}
\item
a descriptive title and the author's name;
\item
an abstract describing the purpose of the experiment
framed as an hypothesis, the experimental methods used,
a summary of the results, and a conclusion that addresses
the hypothesis;
\item
an introduction describing the purpose of the experiment and the details of
the apparatus, including at least one figure, and more where that would be
helpful;
\item
a description of the theoretical model of the experiment and how the
measurements are interpreted;
\item
experimental data presented in a readily understandable form such as tables
and graphs, together with errors;
\item
a description of how you took the raw data and analyzed them in terms of the
theoretical model described earlier, with an emphasis on error propagation;
\item
presentation of the final results and comparison with prior published results
from other experiments, and a conclusion that addresses the hypothesis.
\end{enumerate}

The abstracts should be 150-200 words, and must contain the following elements
summarizing the report:
\begin{enumerate}
\item
a concise statement of the hypothesis to be tested or the quantity measured
in the experiment;
\item
a brief description of the experimental apparatus and the theoretical model in 
words (no equations);
\item
statement of the quantitative results obtained, with values, uncertainties,
and units specified;
\item
conclusions that address the hypothesis.
\end{enumerate}
Reports with abstracts must be submitted in electronic form in pdf format 
sing the submission tool provided on the course web site. The Teaching Assistant
(TA) in charge of your section will explain the schedule for when these reports
are due, and will be responsible for the grading of the regular reports.  The
course instructor will monitor these reports and give feedback to the students
on their titles and abstracts.

\item
{\bf Full-length articles:} Instead of in-class midterm and final
exams, students will chose one of their regular reports from the first
half of the semester, and another from the second half, and expand them
into full-length scientific articles edited into a form that could be
submitted to a scientific journal.  These reports will follow a similar
format to the one described above for the regular reports, but must be
more detailed and complete.  In particular, the introduction must be
separated from the description of the experimental setup and expanded
to include a reference to at least one scientific article, together
with a description of how it is related to the present work.  Whereas
regular reports will normally be only 5-6 pages long, the full-length
articles will normally be 8-10 pages long and consist of 1500-2000
words plus equations, figures, and tables.
Midterm and final articles will be submitted first in draft form,
and then in final form, after feedback has been received from the 
instructor and changes incorporated into the report.  The drafts
must be complete articles; <i>mere skeleton articles, or ones
that are missing significant portions of data or analysis results
will not be accepted as completed drafts.</i> Full-length articles
must be submitted in the form of a pdf file through the submission
tool provided on the huskyct web site.  The pdf file <i>must</i>
contain as its first page a cover letter from the student, followed
by an empty page for instructor feedback and evaluation, followed
by the article.  A template cover letter for draft and final articles
will be provided by the instructor.  The due dates for the
various drafts and final article submission will be announced in
class, and posted on huskyct.

\item
{\bf Final presentations:} Each student will prepare an oral
presentation of approximately 10 minutes duration on the same subject
as their final article.  These presentations will be made to 
other members of the class at the time of the regular lab period
during the last week of classes.  Presentations must be accompanied
by slides in electronic form that describe the apparatus and data
collected in graphical form, and describe the methods used to analyze
the data and the accompanying errors.  The presentation must also
make reference to related measurements that are published in
scientific journals.  The presentation must end with conclusions
that can be drawn from the experimental results.  Because of the
group nature of the final presentations, no late work will be
accepted for the final presentation.
\end{itemize}

\section{Grading policies}

\hspace*{2cm}
\begin{tabular}{lr}
\multicolumn{2}{l}{scoring rubrik for each report} \\
\hline
Content: experimental method & 10\% \\
Content: theory (principles, math) & 10\% \\
Content: quality of results (data, figures) & 20\% \\
Content: analysis (including errors) & 30\% \\
Content: conclusions (soundness, summary) & 10\% \\
Presentation: writing (clarity, correctness) & 20\% \\
\hline
\end{tabular}
\hspace{1cm}
\begin{tabular}{lr}
\multicolumn{2}{l}{breakdown of final grade} \\
\hline
Regular reports & 35\% \\
In-class assignments & 10\% \\
Midterm paper & 10\% \\
Final paper & 25\% \\
Oral presentation & 10\% \\
Lab and numerical skills & 10\% \\
\hline
\end{tabular}
\vspace{1cm}

To pass this course your aggregate score under all of the above
categories must be above a minimum acceptable level.  The two
full-length articles that go through a complete revision and
editing process, considered by themselves apart from the other
graded components of the course, must also be above a passing
grade in order to pass this course.

For students who miss a lab during the semester, one week is
reserved for make-up close to the end of the semester.  Regular
Lab periods take place 2:00-5:00 on Wednesdays.  Groups needing
extended time to complete a measurement should arrange with
the TA to gain access to the laboratory outside of these hours.
Regular reports are normally due at the beginning of the laboratory
period a week after the lab is performed.  Special deadlines for
the full-length articles
will be posted on the huskyct web site.  Work submitted after the
deadline will suffer a 1\% per hour deduction in points, unless
granted an exception by the instructor and/or TA for exceptional
circumstances.  If the exceptional circumstances are known in
advance, the student should contact the TA and/or instructor to
obtain an adjusted deadline to make sure no points are lost.
Students who fail to complete any of the labs or submit weekly
reports will receive a zero for that lab, but the labs with long
reports that substitute for midterm and final exams are mandatory.
You cannot pass this course without completing those assignments.

\section{Academic Integrity}
In this course we aim to conduct ourselves as a community of scholars,
recognizing that academic study is both an intellectual and ethical enterprise.
You are encouraged to build on the ideas and texts of others; that is a vital
part of academic life. You are also obligated to document every occasion when
you use another's ideas, language, or syntax. You are encouraged to study
together, discuss readings outside of class, share your drafts during peer
review and outside of class, and go to the Writing Center with your drafts.
In this course, those activities are well within the bounds of academic honesty.
However, when you use another's ideas or language, whether through direct
quotation, summary, or paraphrase, you must formally acknowledge that debt by
signaling it with a standard form of academic citation. Even one occasion of
academic dishonesty, large or small, on any assignment, large or small, will
result in failure for the entire course and referral to Student Judicial
Affairs. For University policies on academic honesty please see UConn's
Responsibilities of Community Life: 
\url{http://community.uconn.edu/the-student-code-preamble/}.
Please note that ignorance of prevailing academic conventions or of
UConn's policies never excuses a violation. You are encouraged to come see
the instructor if you have questions about when and how to cite; you would
also be wise to consult a writing handbook.

\section{Students With Disabilities}
Students who think that they may need accommodations because of a disability
are encouraged to meet with their instructor privately early in the semester.
Students should also contact the Center for Students with Disabilities as
soon as possible to verify their eligibility for reasonable accommodations.
For more information, please go to \url{http://www.csd.uconn.edu/}.

\section{Help with Writing}
Students are invited to visit the University Writing Center for individualized
tutorials. The Writing Center staff includes talented and welcoming graduate
and undergraduate students from across the humanities, social sciences, and
sciences. They work with writers at any stage of the writing process, from
exploring ideas to polishing final drafts. Their first priority is guiding 
each student's revisions, so they frequently provide a sounding board for a
writer's ideas, arguments, analytical moves, and uses of evidence. They can
also work with you on sentence-level concerns, but please note that they will
not proofread for you; instead, they will help you become a better editor of
your own work. You should come with a copy of the assignment you are working 
on, a current draft (or notes if you are not yet at the draft stage), and
ideas about what you want out of a session. Tutorials run 45 minutes and are
free. You can drop in or make an appointment. For hours, locations, and more
information, please go to \url{http://writingcenter.uconn.edu}.

\section{Contact information}
\begin{itemize}
\item
Prof. Richard Jones, richard.t.jones@uconn.edu, office P-411, office hours
by appointment, please email.
\end{itemize}

\end{document}
%%
%% ****** End of file template.aps ******
%
%
